\section{Inroduction}

\textbf{Cyclic Redundancy Check (CRC)}\\

One of the most common, and one of the most powerful, error-detecting codes is the
cyclic redundancy check (CRC), which can be described as follows. Given a k-bit
block of bits, or message, the transmitter generates an sequence, known
as a frame check sequence (FCS), such that the resulting frame, consisting of n bits,
is exactly divisible by some predetermined number. The receiver then divides the
incoming frame by that number and, if there is no remainder, assumes there was no
error.3
To clarify this, we present the procedure in three equivalent ways:

\begin{itemize}
	\item Modulo 2 arithmetic
	\item Polynomials
	\item Digital logic
\end{itemize}

The methods seen in class where the first two.\\


\textbf{Modulo 2 arithmetic}\\

Modulo 2 arithmetic uses binary addition with no carries,
which is just the \textbf{ exclusive-OR (XOR)} operation. Binary subtraction with no carries
is also interpreted as the XOR operation.

Now define

\begin{equation*}
	T = 2n-kD + F
\end{equation*}

Where:
$P =$ pattern of $n - k + 1$ bits; this is the predetermined divisor\\
$F = (n - k)$-bit FCS, the last $(n - k) $bits of $T$\\
$D =$ $k-$bit block of data, or message, the first $k $bits of $T$\\
$T$= $n$-bit frame to be transmitted