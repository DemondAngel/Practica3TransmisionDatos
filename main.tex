\documentclass{IEEEtran} 
\usepackage[utf8]{inputenc} %utf8 text character encoding
\usepackage[T1]{fontenc}
\usepackage{CJKutf8}
\usepackage{cite}
\usepackage{amsmath,amssymb,amsfonts,nccmath}
\usepackage{hyperref}
\hypersetup{colorlinks,linkcolor={blue},citecolor={blue},urlcolor={red}}  
\usepackage{cleveref} %Para usar crefrange
\usepackage{algorithm,algorithmic}
\usepackage{fancybox, graphicx}
\usepackage{textcomp}
\usepackage[font=footnotesize,labelfont=bf]{caption}
\usepackage[font=footnotesize,labelfont=bf]{subcaption}
\usepackage{graphicx}
\usepackage{color}
\usepackage{gensymb}
\usepackage{dblfloatfix}
\usepackage{lineno}
%\usepackage{biblatex}
\usepackage{enumitem}
\usepackage{wrapfig}
\usepackage{tikz}
\usepackage[autostyle]{csquotes}
\usepackage{listings}

\definecolor{backcolour}{rgb}{0.95,0.95,0.92}
\definecolor{codegreen}{rgb}{0,0.6,0}
\definecolor{codegray}{rgb}{0.5,0.5,0.5}
\definecolor{codepurple}{rgb}{0.58,0,0.82}



\newcommand{\squeezeup}{\vspace{-3mm}}
\usepackage{latexsym} 
\renewcommand{\arraystretch}{1.3}

\title{Practice 3: Signal Analysis: ASK simulation in Matlab}

\author{Alemón Pérez Alejandro, Álvarez Zamora Óscar Eduardo, Gallegos Ruiz Diana Abigail, Rojas Gómez Ian } 

\lstdefinestyle{mystyle}{
	backgroundcolor=\color{backcolour!19},   
	commentstyle=\color{codegreen},
	keywordstyle=\color{blue},
	numberstyle=\tiny\color{codegray},
	stringstyle=\color[rgb]{0.639,0.082,0.082}\ttfamily,
	basicstyle=\ttfamily\footnotesize,
	breakatwhitespace=false,         
	breaklines=true,                 
	captionpos=b,                    
	keepspaces=true,                 
	numbers=left,                    
	numbersep=5pt,                  
	showspaces=false,                
	showstringspaces=false, 
	showtabs=false,                  
	tabsize=2
}

\lstset{style=mystyle}
\begin{document}
	
	\maketitle
	\begin{abstract}
	In this practice we shall implement a CRC error detection with multiple polinomies P(x) in order to detect errors in a telecommunication system based on ARDUINO.
	\end{abstract}10
	\section{Motivation}	
		It is important to understand and identify the components in each of the layers of the OSI model as it is the model of reference from long ago. Thus, it is really necessary to know the functions and applications of each layers not only theoretically but in real practices, an example of these components is the CRC that is used constantly in telecommunications systems and standards.
	\input{arduino.tex}
	\section{Objetives}
		\begin{itemize}
			
			\item 	Implement a CRC (Cyclic Redundancy Check) in the basic 433MHZ ASK RF telecommucation system done in practice 2.3
			
		\end{itemize}

	\section{Inroduction}

\textbf{Cyclic Redundancy Check (CRC)}\\

One of the most common, and one of the most powerful, error-detecting codes is the
cyclic redundancy check (CRC), which can be described as follows. Given a k-bit
block of bits, or message, the transmitter generates an sequence, known
as a frame check sequence (FCS), such that the resulting frame, consisting of n bits,
is exactly divisible by some predetermined number. The receiver then divides the
incoming frame by that number and, if there is no remainder, assumes there was no
error.3
To clarify this, we present the procedure in three equivalent ways:

\begin{itemize}
	\item Modulo 2 arithmetic
	\item Polynomials
	\item Digital logic
\end{itemize}

The methods seen in class where the first two.\\


\textbf{Modulo 2 arithmetic}\\

Modulo 2 arithmetic uses binary addition with no carries,
which is just the \textbf{ exclusive-OR (XOR)} operation. Binary subtraction with no carries
is also interpreted as the XOR operation.

Now define

\begin{equation*}
	T = 2n-kD + F
\end{equation*}

Where:
$P =$ pattern of $n - k + 1$ bits; this is the predetermined divisor\\
$F = (n - k)$-bit FCS, the last $(n - k) $bits of $T$\\
$D =$ $k-$bit block of data, or message, the first $k $bits of $T$\\
$T$= $n$-bit frame to be transmitted
	\section{Develop}
\subsection{Implementing CRC code in Transmmitter}
The code that we had developed before, we adding some new code to add the FCS to our message.

\lstinputlisting[language=Arduino]{Arduino\_Codes/TX\_h\_433P2/Tx\_h\_433P2.ino}

\subsection{Implementing CRC code in Receptor}

Also, we adding CRC detection to our's receptor code.

\lstinputlisting[language=Arduino]{Arduino\_Codes/Rx\_h\_433P2/Rx\_h\_433P2.ino}
\subsection{Testing transmission}
This are the results for the CRC correction seen on Arduino console.


\begin{figure}[!htbp]
	\centering
	\begin{tabular}{cc}
		\begin{subfigure}{.2\textwidth}
			\includegraphics[width=2cm]{images/10cm_p1.png}
			\subcaption{Testing P=1001.}
		\end{subfigure} &
		\begin{subfigure}{.2\textwidth}
			\includegraphics[width=2cm]{images/10cm_p2.png}
			\subcaption{Testing P=11001}
		\end{subfigure}
	\end{tabular}
	\caption{Distance 10cm}
\end{figure}

\newpage
\begin{figure}  [!htbp]
	\centering
	\begin{tabular}{cc}
			\begin{subfigure}{.2\textwidth}
				\includegraphics[width=3cm]{images/50cm_p1.png}
				\subcaption{Testing P=1001.}
			\end{subfigure} &
			\begin{subfigure}{.2\textwidth}
				\includegraphics[width=3cm]{images/50cm_p2.png}
				\subcaption{Testing P=11001}
			\end{subfigure}
	\end{tabular}
	\caption{Distance 50cm}
\end{figure}

\begin{figure}[!htbp]
	\centering
	\begin{tabular}{cc}
		\begin{subfigure}{.2\textwidth}
			\includegraphics[width=3cm]{images/1m_p1.png}
			\subcaption{Testing P=1001.}
		\end{subfigure} &
		\begin{subfigure}{.2\textwidth}
			\includegraphics[width=3cm]{images/1m_p2.png}
			\subcaption{Testing P=11001}
		\end{subfigure}
	\end{tabular}
	\caption{Distance 1m}
\end{figure}

\begin{figure}[!htbp]
	\centering
	\begin{tabular}{cc}
		\begin{subfigure}{.2\textwidth}
			\includegraphics[width=3cm]{images/5m_p1.png}
			\subcaption{Testing P=1001.}
		\end{subfigure} &
		\begin{subfigure}{.2\textwidth}
			\includegraphics[width=3cm]{images/5m_p2.png}
			\subcaption{Testing P=11001}
		\end{subfigure}
	\end{tabular}
	\caption{Distance 5m}
	\label{fig:test5m}
\end{figure}

\newpage
\begin{figure}[!htbp]
	\centering
	\begin{tabular}{cc}
		\begin{subfigure}{.2\textwidth}
			\includegraphics[width=3cm]{images/10m_p1.png}
			\subcaption{Testing P=1001.}
		\end{subfigure} &
		\begin{subfigure}{.2\textwidth}
			\includegraphics[width=3cm]{images/10m_p2.png}
			\subcaption{Testing P=11001}
		\end{subfigure}
	\end{tabular}
	\caption{Distance 10m}
	\label{fig:test10m}
\end{figure}

\begin{figure}[!htbp]
	\centering
	\begin{tabular}{cc}
		\begin{subfigure}{.2\textwidth}
			\includegraphics[width=3cm]{images/15m_p1.png}
			\subcaption{Testing P=1001.}
		\end{subfigure} &
		\begin{subfigure}{.2\textwidth}
			\includegraphics[width=3cm]{images/15m_p2.png}
			\subcaption{Testing P=11001}
		\end{subfigure}
	\end{tabular}
	\caption{Distance 15m}
\end{figure}

\begin{figure} [!htbp]
	\centering
	\begin{tabular}{cc}
		\begin{subfigure}{.2\textwidth}
			\includegraphics[width=3cm]{images/20m_p1.png}
			\subcaption{Testing P=1001.}
		\end{subfigure} &
		\begin{subfigure}{.2\textwidth}
			\includegraphics[width=3cm]{images/20m_p2.png}
			\subcaption{Testing P=11001}
		\end{subfigure}
	\end{tabular}
	\caption{Distance 20m}
	\label{fig:test20m}
\end{figure}

The previous results show us, that the CRC error detection are correct, and the polynomials that we implemented are efective against medium effects. We can see in Figure \ref{fig:test20m} that $P=1001$ detect less corrupted packages than $P=11001$.Also, we received less packages while we were increasing the distance, this showed between Figure \ref{fig:test5m} and Figure \ref{fig:test10m} where we receive more than 500 packages while the other case whe received less than 150 packages, and we could detect more errors.

\subsection{Frame Error Rate}

For Frame Error Rate we have de following graph:
\begin{figure}[!htbp]
	\centering
	\includegraphics [width=9.5cm]{images/FERt.png}
	\caption{FER graph}
\end{figure} 

We can observe that as the distance increases, the FER also increases, this is because less packets arrive and most of them were incorrect.

The values taken were:


	\begin{table} [!htbp]
		\centering
		\caption{FER values}
			\shadowbox{
			\begin{tabular}{|c|c|c|}
			\hline
			\textbf{Distance} & Polynomial 1 & Polynomial 2 \\ \hline
			10 cm    &  	0.2857   & 0.3320    \\
			50 cm  &  0.7849  & 0.3011 \\
			1 m  & 0.6835  & 0.4510 \\
			5 m  & 0.4839 &  0.5869 \\
			10 m  &  0.8980 &  0.7963 \\
			25 m  & 0.9459  & 0.6591\\
			20 m &  0.9615 &     0.6875 \\ \hline
		\end{tabular}
	}
	\end{table}


\newpage
\subsection{FRIIS Equation and Shannon-Hartley Capacity}
\subsubsection{FRIIS Equation}
We use FRIIS equation to calculate the \textbf{reception's power}, we used the following parameters:

\begin{align*}
	G_{Tx}= 3 dBi\\
	G_{Rx}= 3 dBi\\
	c = 3 \times 10^8 \ \left[\frac{m}{s}\right] \\
	fc = 433 MHz\\
	P_{Tx} = 40 dBm\\
	Sensivity = -106 dBm
\end{align*}

And FRIIS equation:

\begin{equation}
	P_{Rx} = P_{Tx} + G_{Tx} + 20l\log_{10}\left(\frac{c}{fc}\right) -20\log(4\pi r)
\end{equation}

We got the following results:

\begin{figure}[!htbp]
	\centering
	\includegraphics[width=9.5cm]{images/friis.png}
	\caption{Result of $P_{Tx}$}
\end{figure}

As we can see, our receptor could receive transmissions along the testing distance because of the fact the sensivity was $-106 dBm$. However during the transmission, it was show how the channel disturbed our information while it was travelling.

\subsubsection{Shannon-Hartley Capacity}
We use Shannon-Hartley equation to calculate Channel's Capacity. We consider a bandwidth of $25 kHz$ because this is the global use of bandwidth, this might change in some countries. \cite{teleradio_2020}

Shannon-Hartley Capacity Equation:
\begin{equation}
	C = B\log_{2}\left(1+\frac{S}{N}\right)\\
\end{equation}

We got the following results:

\begin{figure}[!htbp]
	\centering
	\includegraphics[width=9.5cm]{images/shannon_capacity.png}
	\caption{Shannon Capacity.}
	\label{fig:shannon}
\end{figure}

Figure \ref{fig:shannon}, show us a we had expected, while the SNR is increasing, the channel capacity increases. 
In this practice we noticed that this occurs both, when separating the transmisor and receptor, and because  of  other signals interfering like WIFI, LTE, 5G, microwaves, walls, etc. 


	\section{Conclusions}

\ref{stallings_2000}
\ref{tele radio_2020}

%\printbibliography
	
\end{document}